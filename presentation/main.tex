%%%%%%%%% Notes %%%%%%%%%

%%%%%%%%% Packages %%%%%%%%%
\documentclass[aspectratio=43]{beamer}
\usepackage[T1]{fontenc}
%\usepackage{derivative}
\usepackage{pgfpages}

%%%%%%%%% Packages setup %%%%%%%%%
\setbeameroption{show notes on second screen}
%\setbeameroption{show only notes}
%\setbeamerfont{note page}{size=\tiny}
\setbeamercolor{note page}{bg=white, fg=black}
\setbeamercolor{note title}{bg=white!99!black, fg=black}
\usetheme{Hannover}
\usecolortheme{spruce}
\graphicspath{{./figures/}{../code/figures/}}

%%%%%%%%% Document informations %%%%%%%%%
\title{Summary of \emph{A tutorial on the free-energy framework for modelling perception and learning} by Rafal Bogacz}
\author{Marco Casari}
\date[12/12/2023]{Complex system in neuroscience, 12 December 2023}
\institute[UniTo]{University of Turin}

%%%%%%%%% Document %%%%%%%%%
\begin{document}
\begin{frame}
  \titlepage
  \note{
    \begin{itemize}
      \item % MC general overview of presentation.
    \end{itemize}
  }
\end{frame}



\section{Introduction}
\begin{frame}{Introduction}
  \begin{itemize}
    \item<1-> Rao and Ballard's predictive coding model.
    \note[item]<1->{Prior predictions are compared to stimuli and the model parameters are updated considering prediction errors.}
    \note[item]<1->{Features corresponding to receptive fields in the the primary sensory cortex are learned.}
    \item<2-> Friston's free-energy model.
    \note[item]<2->{}
    \item<3-> Hebbian learning
    \note[item]<1->{Synaptic strenght is changed proportionally to pre-synaptic neuron and post-synaptic neuron activities.}
  \end{itemize}
\end{frame}



\section{Single variable model}
\begin{frame}{Exact solution of the inference problem}
  \begin{itemize}
    \item % MC continue.
  \end{itemize}
\end{frame}

% MC insert Exercise_1.

\begin{frame}{Approximated solution of the inference problem}
  \begin{itemize}
    \item % MC continue.
  \end{itemize}
\end{frame}

% MC insert Exercise_2.

\begin{frame}{Neural implementation}
  \begin{itemize}
    \item % MC continue.
  \end{itemize}
\end{frame}

% MC insert Exercise_3.

% MC here talk about prediction errors.
\begin{frame}{Learning model parameters}
  \begin{itemize}
    \item % MC continue.
  \end{itemize}
\end{frame}

\begin{frame}{Learning relation between variable and stimulus}
  \begin{itemize}
    \item % MC continue.
  \end{itemize}
\end{frame}

\begin{frame}{Free energy framework}
  \begin{itemize}
    \item % MC continue.
  \end{itemize}
\end{frame}



% MC here hint to the use of matrix algebra.
\section{Multiple variables model}
\begin{frame}{Multiple variables model}
  \begin{itemize}
    \item % MC continue.
  \end{itemize}
\end{frame}

% MC here talk about prediction errors.
\begin{frame}{Learning parameters}
  \begin{itemize}
    \item % MC continue.
  \end{itemize}
\end{frame}

\begin{frame}{Hierarchical structure implementation}
  \begin{itemize}
    \item % MC continue.
  \end{itemize}
\end{frame}

% MC here just hint to the possibility of extending local plasticity to mutliple layers (cf. 5.2).
\begin{frame}{Recover local plasticity}
  \begin{itemize}
    \item % MC continue.
  \end{itemize}
\end{frame}

% MC insert Exercise_5.



% MC here resume the Discussion.
\section{Conclusion}
\begin{frame}{Conclusion}
  \begin{itemize}
    \item % MC complete with resume of content and future work.
  \end{itemize}
%  \vfill
%  \onslide<2->{
%    \centering
%    \huge
%    Thank you
%  }
\end{frame}
\end{document}
